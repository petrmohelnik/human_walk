\chapter{Obsah CD}
\paragraph{Adresář \texttt{tex}}
Obsahuje zdrojový tvar písemné zprávy v \LaTeX u. Překlad je možný pomocí \texttt{\$make}. Případně v Texmakeru pomocí pdflatex + bibtex + pdflatex + pdflatex. Adresář \texttt{fig} obsahuje obrázky použité v písemné zprávě.

\paragraph{\texttt{proceduralni\_animace\_lidske\_chuze.pdf}}
Přeložená písemná zpráva ve formátu pdf.

\paragraph{Adresář \texttt{human\_walk}}
Obsahuje solution pro Visual Studio 2013 se zdrojovými kódy aplikace. Ve Visual studiu se aplikace přeloží pomocí Build/Build Solution (F7) a spustí pomocí Debug/Start Debugging (F5). Aplikace vyžaduje podporu OpenGL 3.1. Zdrojové kódy jsou v podadresáři \texttt{human\_walk/source}. V rámci práce nebyly vytvořeny soubory \texttt{imconfig}, \texttt{imgui}, \texttt{imgui\_demo}, \texttt{imgui\_draw}, \texttt{imgui\_impl\_sdl\_gl3}, \texttt{imgui\_internal}, \texttt{stb\_rect\_pack}, \texttt{stb\_textedit}, \texttt{stb\_truetype} a \texttt{lodepng}.

\paragraph{Adresář \texttt{Release}}
Obsahuje přeloženou aplikaci pro Windows spustitelnou pomocí \texttt{human\_walk.exe} a další soubory potřebné pro její běh.

\paragraph{\texttt{video.mp4}}
Video s prezentací projektu. Dostupné také z \url{https://youtu.be/Z5Qckdr9yDI}

\chapter{Manuál}

\paragraph{Ovládání aplikace}
\begin{itemize} \itemsep1pt \parskip0pt \parsep0pt
\item[] \textbf{L} - vypnutí/zapnutí úrovně detailu
\item[] \textbf{F} - přepínání mezi flat a smooth stínováním
\item[] \textbf{B} - vypnutí/zapnutí skyboxu
\item[] \textbf{T} - vypnutí/zapnutí terénu
\item[] \textbf{C} - zapnutí/vypnutí zobrazení křivky definující dráhu
\item[] \textbf{V} - zapnutí/vypnutí zobrazení rozdělení křivky na segmenty
\item[] \textbf{K} - zapnutí/vypnutí barevného odlišení úrovní detailu
\item[] \textbf{E} - zapnutí/vypnutí zobrazení dráhy jako drátového modelu
\item[] \textbf{N} - spustí normovaný průchod
\item[] \textbf{O} - zapnutí/vypnutí zobrazení dráhy pomocí přechodů mezi úrovněmi. Od nejvyšší úrovně k nejnižší a zpět k nejvyšší, bez ohledu na pozici kamery.
\item[] \textbf{$\uparrow$ / $\downarrow$} - pohyb po dráze vpřed/vzad
\item[] \textbf{WSAD} - volný pohyb kamerou
\item[] \textbf{+/-} - oddálení/přiblížení kamery
\item[] \textbf{2/8} - rotace kamery okolo osy X
\item[] \textbf{4/6} - rotace kamery okolo osy Y
\item[] \textbf{SHIFT} - Urychlí pohyb kamery při souběžném držení s klávesou pro pohyb kamery 
\end{itemize}

\paragraph{Změna parametrů aplikace}
Parametry je možné měnit v souboru \texttt{params.txt}. Je nutné zadat všechny níže popsané parametry. Tvar je \textbf{parametr=hodnota} bez mezer.
\begin{itemize} \itemsep1pt \parskip0pt \parsep0pt
\item[] \textbf{SEED} - seed pro generování náhodných čísel
\item[] \textbf{MAP\_SIZE} - velikost mapy
\item[] \textbf{MAP\_VERTICES} - počet vrcholů mapy
\item[] \textbf{SPEED} - rychlost pohybu kamery po dráze
\item[] \textbf{TRIANGLES\_1} - počet trojúhelníků původního modelu
\item[] \textbf{TRIANGLES\_2/3/4} - počet trojúhelníků pro vytvoření nižších úrovní detailu
\item[] \textbf{MAX\_TRIANGLES} - maximální počet trojúhelníků při použití úrovně detailu
\item[] \textbf{INCREMENT\_1/2/3} - inkrementy pro určení intervalů vzdáleností jednotlivých úrovní detailu
\item[] \textbf{N} - počet oblastí v jednom kvadrantu pro generování dráhy
\end{itemize}

%\chapter{Manual}
%\chapter{Konfigrační soubor}
%\chapter{RelaxNG Schéma konfiguračního soboru}
%\chapter{Plakat}

