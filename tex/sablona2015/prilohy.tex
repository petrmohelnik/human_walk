\chapter{Obsah CD}
\begin{itemize}

\item \textbf{tex/ - } 
Obsahuje zdrojový tvar písemné zprávy v \LaTeX u. Překlad je možný pomocí \texttt{\$make}. Případně v Texmakeru pomocí pdflatex + bibtex + pdflatex + pdflatex. Adresář \texttt{fig/} obsahuje obrázky použité v písemné zprávě.

\item \textbf{proceduralni\_animace\_lidske\_chuze.pdf - }
Přeložená písemná zpráva ve formátu pdf.

\item \textbf{human\_walk/ - }
Obsahuje \textit{solution} pro Visual Studio 2013 se zdrojovými kódy aplikace. Ve Visual studiu se aplikace přeloží pomocí Build/Build Solution (ctrl + F7) a spustí pomocí Debug/Start Debugging (F5). Aplikace vyžaduje podporu OpenGL 3.1. Zdrojové kódy jsou v podadresáři \texttt{human\_walk/source/}. V rámci práce nebyly vytvořeny soubory \texttt{imconfig}, \texttt{imgui}, \texttt{imgui\_demo}, \texttt{imgui\_draw}, \texttt{imgui\_impl\_sdl\_gl3}, \texttt{imgui\_internal}, \texttt{stb\_rect\_pack}, \texttt{stb\_textedit}, \texttt{stb\_truetype} a \texttt{lodepng}.

\item \textbf{Release/ - }
Obsahuje přeloženou aplikaci pro Windows a další soubory potřebné pro její běh.

\item \textbf{video.mp4 - }
Video s prezentací projektu. Dostupné také z \url{https://youtu.be/Z5Qckdr9yDI}

\item \textbf{skeleton.dae - }
Kostra nad kterou aplikace pracuje.

\end{itemize}

\chapter{Manuál}
Po načtení aplikace se zobrazí první model na nerovném terénu. Pro změnu parametrů je možné použít grafické uživatelské rozhraní, jehož okna je možné skrýt dvojklikem na jejich lišty. Pomocí tlačítka \texttt{Next terrain} je možné přepínat mezi nerovným, schodovitým a rovným terénem. Tlačítko \texttt{Next model} přepne na další scénu s  jiným modelem.
Rotace kamery se ovládá pravým tlačítkem myši. Kolečkem myši je možné přibližovat a oddalovat kameru. Kamera se pohybuje spolu s modelem.
Některé funkce je také možné ovládat pomocí tlačítek na klávesnici: 
\begin{itemize} \itemsep1pt \parskip0pt \parsep0pt
\item[] \textbf{mezerník} - pozastavení/spuštění animace
\item[] \textbf{tabulátor} - přepínání mezi scénami s různými modely
\item[] \textbf{1} - vypnutí/zapnutí zobrazení kostry
\item[] \textbf{2} - vypnutí/zapnutí zobrazení modelu
\item[] \textbf{+} - přiblížení kamery
\item[] \textbf{-} - oddálení kamery
\end{itemize}

Pro připravení modelu pro běh v aplikaci je potřeba k němu připojit přiloženou kostru např. pomocí automatického přiřazení vah a vyexportovat v Blenderu do formátu COLLADA.

%\chapter{Manual}
%\chapter{Konfigrační soubor}
%\chapter{RelaxNG Schéma konfiguračního soboru}
%\chapter{Plakat}

