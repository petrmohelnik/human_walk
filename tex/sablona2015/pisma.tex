
%---rm---------------
\renewcommand{\rmdefault}{lmr}%zavede Latin Modern Roman jako rm
%---sf---------------
\renewcommand{\sfdefault}{qhv}%zavede TeX Gyre Heros jako sf
%---tt------------
\renewcommand{\ttdefault}{lmtt}% zavede Latin Modern tt jako tt

%-----------------------------------------------------------------------------


% Níže jsou deklarace fontů pro testování a ladění JVS 
% - doporučuje se NEPOUŽÍVAT
% Deklarace nejsou doladěné !!!
% Times New Roman není dle JVS povolený (je tu na ukázku)
%-----------------------------------------------------------------------------

\ifOPEN
  \pdfmapfile{=OpenSansfontspdf.map}
  \DeclareFontFamily{T1}{OpenSans}{}
  \DeclareFontShape{T1}{OpenSans}{b}{n}{<->recOpenSans-Bold}{}
  \DeclareFontShape{T1}{OpenSans}{b}{it}{<->recOpenSans-BoldItalic}{}
  \DeclareFontShape{T1}{OpenSans}{eb}{n}{<->recOpenSans-ExtraBold}{}
  \DeclareFontShape{T1}{OpenSans}{eb}{it}{<->recOpenSans-ExtraBoldItalic}{}
  \DeclareFontShape{T1}{OpenSans}{m}{it}{<->recOpenSans-Italic}{}
  \DeclareFontShape{T1}{OpenSans}{l}{n}{<->recOpenSans-Light}{}
  \DeclareFontShape{T1}{OpenSans}{l}{it}{<->recOpenSans-LightItalic}{}
  \DeclareFontShape{T1}{OpenSans}{m}{n}{<->recOpenSans-Regular}{}
  \DeclareFontShape{T1}{OpenSans}{sb}{n}{<->recOpenSans-Semibold}{}
  \DeclareFontShape{T1}{OpenSans}{sb}{it}{<->recOpenSans-SemiboldItalic}{}
  \renewcommand{\rmdefault}{OpenSans}
  \renewcommand{\sfdefault}{OpenSans}
\else
  \iftoggle{declare_open}{
    \pdfmapfile{=OpenSansfontspdf.map}
    \DeclareFontFamily{T1}{OpenSans}{}
    \DeclareFontShape{T1}{OpenSans}{b}{n}{<->recOpenSans-Bold}{}
    \DeclareFontShape{T1}{OpenSans}{b}{it}{<->recOpenSans-BoldItalic}{}
    \DeclareFontShape{T1}{OpenSans}{eb}{n}{<->recOpenSans-ExtraBold}{}
    \DeclareFontShape{T1}{OpenSans}{eb}{it}{<->recOpenSans-ExtraBoldItalic}{}
    \DeclareFontShape{T1}{OpenSans}{m}{it}{<->recOpenSans-Italic}{}
    \DeclareFontShape{T1}{OpenSans}{l}{n}{<->recOpenSans-Light}{}
    \DeclareFontShape{T1}{OpenSans}{l}{it}{<->recOpenSans-LightItalic}{}
    \DeclareFontShape{T1}{OpenSans}{m}{n}{<->recOpenSans-Regular}{}
    \DeclareFontShape{T1}{OpenSans}{sb}{n}{<->recOpenSans-Semibold}{}
    \DeclareFontShape{T1}{OpenSans}{sb}{it}{<->recOpenSans-SemiboldItalic}{}
  }
\fi


\ifVAFLE
  \pdfmapfile{=Vafle_VUT_fontspdf.map}
  \DeclareFontFamily{T1}{VafleVUT}{}
  \DeclareFontShape{T1}{VafleVUT}{m}{n}{<->recVafle_VUT_Regular}{}
  \DeclareFontShape{T1}{VafleVUT}{b}{n}{<->recVafle_VUT_Bold}{}
  \DeclareFontShape{T1}{VafleVUT}{l}{n}{<->recVafle_VUT_Light}{}
  \renewcommand{\rmdefault}{VafleVUT}
  \renewcommand{\sfdefault}{VafleVUT}
  % Tohle je škaredý hack - Vafle nemá it a když se s tím nic neudělá, kurzíva
  % se nijak neprojeví (jen varováním při překladu). Nicméně "doplňkový" font 
  % OpenSans kurzívu má a v semibold to dle mého názoru vypadá pro demonstrační
  % účely při ladění JVS přijatelně.
  \let\oldit\it
  \renewcommand{\it}{\usefont{T1}{OpenSans}{sb}{it}}
\else
  \ifTVAFLE
    \pdfmapfile{=Vafle_VUT_fontspdf.map}
    \DeclareFontFamily{T1}{VafleVUT}{}
    \DeclareFontShape{T1}{VafleVUT}{m}{n}{<->recVafle_VUT_Regular}{}
    \DeclareFontShape{T1}{VafleVUT}{b}{n}{<->recVafle_VUT_Bold}{}
    \DeclareFontShape{T1}{VafleVUT}{l}{n}{<->recVafle_VUT_Light}{}
    \pdfmapfile{=OpenSansfontspdf.map}
    \DeclareFontFamily{T1}{OpenSans}{}
    \DeclareFontShape{T1}{OpenSans}{b}{n}{<->recOpenSans-Bold}{}
    \DeclareFontShape{T1}{OpenSans}{b}{it}{<->recOpenSans-BoldItalic}{}
    \DeclareFontShape{T1}{OpenSans}{eb}{n}{<->recOpenSans-ExtraBold}{}
    \DeclareFontShape{T1}{OpenSans}{eb}{it}{<->recOpenSans-ExtraBoldItalic}{}
    \DeclareFontShape{T1}{OpenSans}{m}{it}{<->recOpenSans-Italic}{}
    \DeclareFontShape{T1}{OpenSans}{l}{n}{<->recOpenSans-Light}{}
    \DeclareFontShape{T1}{OpenSans}{l}{it}{<->recOpenSans-LightItalic}{}
    \DeclareFontShape{T1}{OpenSans}{m}{n}{<->recOpenSans-Regular}{}
    \DeclareFontShape{T1}{OpenSans}{sb}{n}{<->recOpenSans-Semibold}{}
    \DeclareFontShape{T1}{OpenSans}{sb}{it}{<->recOpenSans-SemiboldItalic}{}
  \fi
\fi

\ifARIAL
  \pdfmapfile{=arialfontspdf.map}
  \DeclareFontFamily{T1}{arial}{}
  \DeclareFontShape{T1}{arial}{b}{n}{<->recarialbd}{}
  \DeclareFontShape{T1}{arial}{b}{sl}{<->recarialbdo}{}
  \DeclareFontShape{T1}{arial}{b}{it}{<->recarialbi}{}
  \DeclareFontShape{T1}{arial}{m}{n}{<->recarial}{}
  \DeclareFontShape{T1}{arial}{m}{sl}{<->recarialo}{}
  \DeclareFontShape{T1}{arial}{m}{it}{<->recariali}{}
  \DeclareFontShape{T1}{arial}{bx}{n}{<->ssub * arial/b/n}{}
  \DeclareFontShape{T1}{arial}{bx}{sl}{<->ssub * arial/b/sl}{}
  \DeclareFontShape{T1}{arial}{bx}{it}{<->ssub * arial/b/it}{}
  \renewcommand{\rmdefault}{arial}
  \renewcommand{\sfdefault}{arial}
\else
  \ifTARIAL
    \pdfmapfile{=arialfontspdf.map}
    \DeclareFontFamily{T1}{arial}{}
    \DeclareFontShape{T1}{arial}{b}{n}{<->recarialbd}{}
    \DeclareFontShape{T1}{arial}{b}{sl}{<->recarialbdo}{}
    \DeclareFontShape{T1}{arial}{b}{it}{<->recarialbi}{}
    \DeclareFontShape{T1}{arial}{m}{n}{<->recarial}{}
    \DeclareFontShape{T1}{arial}{m}{sl}{<->recarialo}{}
    \DeclareFontShape{T1}{arial}{m}{it}{<->recariali}{}
    \DeclareFontShape{T1}{arial}{bx}{n}{<->ssub * arial/b/n}{}
    \DeclareFontShape{T1}{arial}{bx}{sl}{<->ssub * arial/b/sl}{}
    \DeclareFontShape{T1}{arial}{bx}{it}{<->ssub * arial/b/it}{}
  \fi
\fi

\ifTIMES
  \pdfmapfile{=timesfontspdf.map}
  \DeclareFontFamily{T1}{times}{}
  \DeclareFontShape{T1}{times}{m}{n}{<->rectimes}{}
  \DeclareFontShape{T1}{times}{m}{it}{<->rectimesi}{}
  \DeclareFontShape{T1}{times}{b}{n}{<->rectimesbd}{}
  \DeclareFontShape{T1}{times}{b}{it}{<->rectimesbi}{}
  \renewcommand{\rmdefault}{times}
  \renewcommand{\sfdefault}{times}
\fi
